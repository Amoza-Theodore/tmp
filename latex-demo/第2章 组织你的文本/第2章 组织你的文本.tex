\documentclass[11pt]{beamer}
\usepackage{ctex}
\usepackage[utf8]{inputenc}
\usepackage[T1]{fontenc}
\usepackage{lmodern}
\usepackage{frame}
\usepackage{fancyvrb}
\usetheme{CambridgeUS}


\begin{document}
	%\author{}
	\title{\LaTeX 入门}
	\subtitle{第2章 组织你的文本}
	%\logo{}
	%\institute{}
	\date{2022年8月3日}
	%\subject{}
	%\setbeamercovered{transparent}
	%\setbeamertemplate{navigation symbols}{}
	\begin{frame}[plain]
		\maketitle
	\end{frame}

\part{文字与符号}

\section{正确使用标点}
	
\begin{frame}[fragile]{文字与符号}{正确使用标点}
\begin{Verbatim}[tabsize=2]
An inter-word dash or, hyphen, as in X-ray.

A medium dash for number ranges, like 1--2.

A punctuation dash---like this. \\

Good: One, two, three\dots
\end{Verbatim}
\end{frame}

\begin{frame}[fragile]{文字与符号}{正确使用标点}
\begin{Verbatim}[tabsize=2]
\# \quad \$ \quad \% \quad \& \quad
\{ \quad \} \quad \_ \quad
\textbackslash

Happy \TeX ing. Happy \TeX\ ing.
Happy \TeX{} ing. Happy {\TeX} ing.
\end{Verbatim}
\end{frame}

\begin{frame}[fragile]{文字与符号}{正确使用标点}
\begin{Verbatim}[tabsize=2]
\begin{align*}
	[2 - (3+5)]\times 7 &= 42 \\[2cm]
	[2 + (3-5)]\times 7 &= 0
\end{align*}
\end{Verbatim}
\end{frame}

\section{特殊符号}

\begin{frame}[fragile]{文字与符号}{特殊符号}
\begin{Verbatim}[tabsize=2]
\usepackage[utf8]{inputenc}

\S \dag \ddag \P \copyright \textregistered 
\texttrademark \pounds \textbullet
\end{Verbatim}
\end{frame}

\section{字体}

\begin{frame}[fragile]{文字与符号}{字体}
\begin{Verbatim}[tabsize=2]
\textbf{This is a paragraph of bold and
\textit{italic font, sometimes returning
to \textnormal{normal font} is necessary.}}
\end{Verbatim}
\end{frame}

\begin{frame}[fragile]{文字与符号}{字体}
\begin{Verbatim}[tabsize=2]
{\songti 宋体} \quad {\heiti 黑体} \quad
{\fangsong 仿宋} \quad {\kaishu 楷书}
\end{Verbatim}
\end{frame}

\begin{frame}[fragile]{文字与符号}{字体}
\begin{Verbatim}[tabsize=2]
You \emph{should} use fonts carefully.

\textit{You \emph{should} use fonts carefully.}

This is {\em emphasized} text.
\end{Verbatim}
\end{frame}

\begin{frame}[fragile]{文字与符号}{字体}
\begin{Verbatim}[tabsize=2]
\newcommand\Emph{\textbf}
This is \Emph{emphasized} text.

\underline{Emphasized} text and \underline{another}.
\end{Verbatim}
\end{frame}

\begin{frame}[fragile]{文字与符号}{字体}
\begin{Verbatim}[tabsize=2]
% 导言区用 \usepackage{ulem}
\uline{Emphasized} text and \uline{another}.
A \emph{very very very very very very very
	very very very very very} long sentence.
\end{Verbatim}
\end{frame}

\begin{frame}[fragile]{文字与符号}{字体}
\begin{Verbatim}[tabsize=2]
% 导言区用 \usepackage[normalem]{ulem}
\uline{Emphasized} text and \uline{another}.
\end{Verbatim}
\end{frame}

\begin{frame}[fragile]{文字与符号}{字体}
\begin{Verbatim}[tabsize=2]
\uuline{urgent}\qquad \uwave{boat}\qquad
\sout{wrong}\qquad \xout{removed}\qquad
\dashuline{dashing}\qquad \dotuline{dotty}
\end{Verbatim}
\end{frame}

\section{字号与间距}

\begin{frame}[fragile]{文字与符号}{字号与间距}
\begin{Verbatim}[tabsize=2]
\tiny \scriptsize \footnotesize \small \normalsize 
\large \Large \LARGE \huge \Huge
\zihao{0}

The text can be {\Large larger}.
\end{Verbatim}
\end{frame}

\begin{frame}[fragile]{文字与符号}{字号与间距}
\begin{Verbatim}[tabsize=2]
\usepackage{setspace}% preamble

\setstretch{2.0}
\singlespacing \onehalfspacing \doublespacing
\end{Verbatim}
\end{frame}

\section{水平间距与盒子}

\begin{frame}[fragile]{文字与符号}{水平间距与盒子}
\begin{tabular}{lp{.9\linewidth}}
pt & point 点(欧美传统排版的长度单位,有时叫做“磅”) \\
pc & pica(1 pc = 12 pt,相当于四号字大小) \\
in & inch 英寸(1 in = 72.27 pt) \\
bp & big point 大点(在 PostScript 等其他电子排版领域的 point 都指大点,1 in = 72 bp) \\
cm & centimeter 厘米(2:54 cm = 1 in) \\
mm & millimeter 毫米(10 mm = 1 cm) \\
dd & didot point(欧洲大陆使用,1157 dd = 1238 pt) \\
cc & cicero(欧洲大陆使用,pica 的对应物,1 cc = 12 dd) \\
sp & scaled point(TEX中最小的长度单位,所有长度都是它的倍数,65536 sp = 1 pt) \\
em & 全身(字号对应的长度,等于一个 \verb|\quad| 的长度,也称为“全方”。本义是大写字母 M 的宽度) \\
ex & x-height(与字号相关,由字体定义。本义是小写字母 x 的高度) \\
\end{tabular}
\end{frame}

\begin{frame}[fragile]{文字与符号}{水平间距与盒子}
\begin{Verbatim}[tabsize=2]
Space\hspace{1cm}1cm

text\\
\hspace{1cm}text\\
\hspace*{1cm}text

left \hspace{\fill} middle \hfill right
left\hspace{\stretch{2}}$2/3$\hspace{\fill}right
\end{Verbatim}
\end{frame}

\begin{frame}[fragile]{文字与符号}{水平间距与盒子}
盒子(box)是 TEX 中的基本处理单位,一个字符、一行文字、一个页面、一张表格在 TEX 中都是一个盒子。
	
\verb|\makebox|[<宽度>][<位置>]{<内容>}
对齐参数可取 c(中)、l(左)、r(右)、s(分散),默认居中。
	
\begin{Verbatim}[tabsize=2]
\makebox[1em]{\textbullet}text \\
\makebox[5cm][s]{some stretched text}
\end{Verbatim}

\end{frame}

\begin{frame}[fragile]{文字与符号}{水平间距与盒子}
\begin{Verbatim}[tabsize=2]
\fbox{framed} \\
\framebox[3cm][s]{framed box}
\end{Verbatim}
\end{frame}

\part{段落与文本环境}

\section{正文段落}

\begin{frame}[fragile]{段落与文本环境}{正文段落}
\begin{Verbatim}[tabsize=2]
\raggedright
% \raggedleft, \centering
English words like ‘technology’ stem from a
Greek root beginning with the letters τεχ\dots
\end{Verbatim}
\end{frame}

\begin{frame}[fragile]{段落与文本环境}{正文段落}
\begin{Verbatim}[tabsize=2]
\begin{center}
	居中
\end{center}
\end{Verbatim}
\end{frame}

\section{文本环境}

\begin{frame}[fragile]{段落与文本环境}{文本环境}
\begin{Verbatim}[tabsize=2]
\begin{abstract}
	本课讲解 \LaTeX{} 的使用。
\end{abstract}

\ctexset{abstractname={摘\quad 要}}
\end{Verbatim}
\end{frame}

\section{列表环境}

\begin{frame}[fragile]{段落与文本环境}{列表环境}
\begin{Verbatim}[tabsize=2]
\begin{enumerate}
	\item 中文
	\item English
	\item Français
\end{enumerate}

\begin{itemize}
	\item 中文
	\item English
	\item Français
\end{itemize}
\end{Verbatim}
\end{frame}

\begin{frame}[fragile]{段落与文本环境}{列表环境}
\begin{Verbatim}[tabsize=2]
\begin{description}
	\item[中文] 中国的语言文字
	\item[English] The language of England
	\item[Français] La langue de France
\end{description}
\end{Verbatim}
\end{frame}

\begin{frame}[fragile]{段落与文本环境}{列表环境}
\begin{Verbatim}[tabsize=2]
\begin{enumerate}
	\item 中文
	\begin{enumerate}
		\item 古代汉语
		\item 现代汉语
	\end{enumerate}
	\item English
	\item Français
\end{enumerate}
\end{Verbatim}
\end{frame}

\begin{frame}[fragile]{段落与文本环境}{列表环境}
\begin{Verbatim}[tabsize=2]
\begin{enumerate}
	\item 中文
	\item[1a.] 汉语
	\item English
\end{enumerate}

\begin{itemize}
	\item[\dag] 中文
	\item English
	\item Français
\end{itemize}
\end{Verbatim}
\end{frame}

\begin{frame}[fragile]{段落与文本环境}{列表环境}
	\begin{Verbatim}[tabsize=2]
% enumi, enumii, enumiii, enumiv
\begin{enumerate}
	\item 这是编号 \theenumi
	\item 这是编号 \theenumi
\end{enumerate}
	\end{Verbatim}
\end{frame}

\begin{frame}[fragile]{段落与文本环境}{列表环境}
	\begin{Verbatim}[tabsize=2]
\begin{enumerate}
\item 编号
	\arabic{enumi}, \roman{enumi}, \Roman{enumi},
	\alph{enumi}, \Alph{enumi}, \fnsymbol{enumi}
\item 编号
	\arabic{enumi}, \roman{enumi}, \Roman{enumi},
	\alph{enumi}, \Alph{enumi}, \fnsymbol{enumi}
\item 编号
	\arabic{enumi}, \roman{enumi}, \Roman{enumi},
	\alph{enumi}, \Alph{enumi}, \fnsymbol{enumi}
\end{enumerate}
	\end{Verbatim}
\end{frame}

\begin{frame}[fragile]{段落与文本环境}{列表环境}
	\begin{Verbatim}[tabsize=2]
\renewcommand\theenumi{\roman{enumi}}
\renewcommand\labelenumi{(\theenumi)}
\begin{enumerate}
	\item 使用中文
	\item Using English
\end{enumerate}
	\end{Verbatim}
\end{frame}

\begin{frame}[fragile]{段落与文本环境}{列表环境}
	\begin{Verbatim}[tabsize=2]
% 计数器设置,通常在导言区
\newcounter{mycnt}
\setcounter{mycnt}{0}
% 默认值就是 0
\renewcommand\themycnt{\arabic{mycnt}}
% 默认值就是阿拉伯数字
% 计数器使用,通常做成自定义命令的一部分
\stepcounter{mycnt}\themycnt 输出计数器值为 1;
\stepcounter{mycnt}\themycnt 输出计数器值为 2;
\addtocounter{mycnt}{1}\themycnt 输出计数器值为 3;
\addtocounter{mycnt}{-1}\themycnt 输出计数器值为 2;
\addtocounter{mycnt}{-1}\themycnt 输出计数器值为 1。
	\end{Verbatim}
\end{frame}

\begin{frame}[fragile]{段落与文本环境}{列表环境}
	\begin{Verbatim}[tabsize=2]
\newcounter{quiz}[section]
\renewcommand\thequiz{\thesection-\arabic{quiz}}
	\end{Verbatim}
\end{frame}

\begin{frame}[fragile]{段落与文本环境}{列表环境}
	\begin{Verbatim}[tabsize=2]
% \usepackage{enumitem}
\begin{enumerate}[label=(\arabic*)]
	\item 中文
	\item English
	\item Français
\end{enumerate}
	\end{Verbatim}
\end{frame}

\section{定理类环境}

\begin{frame}[fragile]{段落与文本环境}{定理类环境}
	\begin{Verbatim}[tabsize=2]
\newtheorem{thm}{定理} % 一般在导言区
\begin{thm}
直角三角形斜边的平方等于两腰的平方和。
\end{thm}

\begin{thm}[勾股定理]
直角三角形斜边的平方等于两腰的平方和。
\end{thm}
	\end{Verbatim}
\end{frame}

\begin{frame}[fragile]{段落与文本环境}{定理类环境}
	\begin{Verbatim}[tabsize=2]
\newtheorem{lemma}{引理}[chapter]% 按章
\begin{lemma}偏序集可良序化。\end{lemma}
\begin{lemma}实数集不可数。\end{lemma}
	\end{Verbatim}
\end{frame}

\begin{frame}[fragile]{段落与文本环境}{定理类环境}
	\begin{Verbatim}[tabsize=2]
\newtheorem{prop}[thm]{命题}
\begin{prop}
直角三角形的斜边大于直角边。
\end{prop}
	\end{Verbatim}
\end{frame}

\begin{frame}[fragile]{段落与文本环境}{定理类环境}
	\begin{Verbatim}[tabsize=2]
% 导言区
\usepackage[thmmarks]{ntheorem}
{
	% 利用分组,格式设置只作用于证明环境
	\theoremstyle{nonumberplain}
	\theoremheaderfont{\bfseries}
	\theorembodyfont{\normalfont}
	\theoremsymbol{\mbox{$\Box$}} % 放进盒子,或用 \ensuremath
	\newtheorem{proof}{证明}
}

\begin{proof}
证明是显然的。
\end{proof}
	\end{Verbatim}
\end{frame}

\section{抄录和代码环境}

\begin{frame}[fragile]{段落与文本环境}{抄录和代码环境}
	\begin{Verbatim}[tabsize=2]
\verb”\LaTeX \& \TeX” \qquad \verb!\/}{#$%&~!

显示空格 \verb*!1 2 3   4!
	\end{Verbatim}
\end{frame}

\begin{frame}[fragile]{段落与文本环境}{抄录和代码环境}
	\begin{Verbatim}[tabsize=2]
\begin{verbatim}
#!usr/bin/env perl
$name = ”guy”;
print ”Hello, $name!\n”;
\end{verbatim}

\begin{verbatim*}
	#include <stdio.h>
	main() {
		printf(”Hello, world.\n”);
	}
\end{verbatim*}
	\end{Verbatim}
\end{frame}

\begin{frame}[fragile]{段落与文本环境}{抄录和代码环境}
	\begin{Verbatim}[tabsize=2]
% \usepackage{fancyvrb}
\SaveVerb{myverb}|#$%^&|
\fbox{套中 \UseVerb{myverb}}
	\end{Verbatim}
\end{frame}

\begin{frame}[fragile]{段落与文本环境}{抄录和代码环境}
	\begin{Verbatim}[tabsize=2]
% 导言区使用 \usepackage{listings}
\lstset{columns=flexible,numbers=left,
	numberstyle=\footnotesize}
\begin{lstlisting} % [language=C]
/* hello.c */
#include <stdio.h>
main() {
	printf(”Hello.\n”);
}
\end{lstlisting} 

% algorithm2e, clrscode
	\end{Verbatim}
\end{frame}

\begin{frame}[fragile]{段落与文本环境}{抄录和代码环境}
	\begin{Verbatim}[tabsize=2]
\begin{tabbing}
格式\hspace{3em} \= 作者 \\
Plain \TeX \> 高德纳 \\
\LaTeX \> Leslie Lamport
\end{tabbing}

\begin{tabbing}
格式\hspace{3em} \= 作者 \kill
Plain \TeX \> 高德纳 \\
\LaTeX \> Leslie Lamport
\end{tabbing}
	\end{Verbatim}
\end{frame}

\section{脚注与边注}

\begin{frame}[fragile]{段落与文本环境}{脚注与边注}
	\begin{Verbatim}[tabsize=2]
例如\footnote{这是一个脚注。}。

\usepackage{pifont}
\renewcommand\thefootnote{
	\ding{\numexpr171+\value{footnote}}}
	\end{Verbatim}
\end{frame}

\part{文档的层次结构}

\section{标题和标题页}

\begin{frame}[fragile]{文档的层次结构}{标题和标题页}
	\begin{Verbatim}[tabsize=2]
\title{杂谈勾股定理\\——勾股定理的历史与现状}
\author{张三\\九章学堂}
\date{庚寅盛夏}

\author{张三\\九章学堂 \and 李四\\天元研究所}

\ctexset{today=small}
\ctexset[today=big]
\ctexset[today=old]

\title{杂谈勾股定理\thanks{本文由九章基金会赞助。}}
\author{张三\thanks{九章学堂讲师。}\\九章学堂}
	\end{Verbatim}
\end{frame}

\section{划分章节}

\begin{frame}[fragile]{文档的层次结构}{划分章节}
	\begin{Verbatim}[tabsize=2]
part, chapter, section, subsection, subsubsection, 
paragraph, subparagraph  % table of book

\chapter*[展望与未来]{展望与未来:畅想新时代的计算机排版软件}
\chapter[展望与未来]{展望与未来:畅想新时代的计算机排版软件}

% ...
\appendix
\chapter{习题解答}
% ...
	\end{Verbatim}
\end{frame}

\section{多文件编译}

\begin{frame}[fragile]{文档的层次结构}{多文件编译}
	\begin{Verbatim}[tabsize=2]
% languages.tex
%
整个文档的主文件
\documentclass{ctexbook}
\title{语言}
\author{张三 \and 李四}
% \includeonly{lang-natual}  % 只编译“自然语言”一章 
\begin{document}
\maketitle
\tableofcontents
\chapter{自然语言}
\chapter{计算机语言}
\end{document}
	\end{Verbatim}
\end{frame}

\begin{frame}[fragile]{文档的层次结构}{多文件编译}
	\begin{Verbatim}[tabsize=2]
% lang-natural.tex
%
“自然语言”一章,不能单独编译
\chapter{自然语言}

% lang-computer.tex
% “计算机语言”一章,不能单独编译
\chapter{计算机语言}
	\end{Verbatim}
\end{frame}

\begin{frame}[fragile]{文档的层次结构}{多文件编译}
	\begin{Verbatim}[tabsize=2]
% main.tex
%
主文档
\documentclass{ctexart}
\input{preamble}
% 复杂的导言区设置
\begin{document}
	……(文档的内容)
\end{document}
	\end{Verbatim}
\end{frame}

\end{document}